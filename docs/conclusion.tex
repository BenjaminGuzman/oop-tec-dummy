\section{Conclusión personal}
Concluyo que para el desarrollo de aplicaciones gráficas Python (y tkinter) no es buena idea, por lo menos para aplicaciones que requieran una interfaz bien diseñada y con elementos robustos, pues hay otras alternativas mucho más robustas que ofrecen prácticamente lo mismo, sin embargo, creo que para la parte lógica del programa es bastante bueno pues la escritura de su código es similar a lo que sería el pseudo código (en inglés, claro), además de que tiene muchas funcionalidades ya incluidas en el propio lenguaje, pero esto también conlleva mucho más trabajo por parte del desarrollador si se quiere hacer una aplicación rápida y robusta pues por un lado Python es interpretado y a diferencia de otros lenguajes que también lo son (e. g. JavaScript) Python es mucho más lento y además se debe tener mucho cuidado con los tipos lo cual puede llegar a dificultar un poco más la escritura tanto del código como la de su documentación.