\documentclass{article}

\usepackage[margin=1cm]{geometry}
\newcommand{\dbtable}{\subsection}
\newcommand{\mono}{\texttt}

\usepackage{listings}
\usepackage{xcolor}

\colorlet{punct}{red!60!black}
\definecolor{background}{HTML}{EEEEEE}
\definecolor{delim}{RGB}{20,105,176}
\colorlet{numb}{magenta!60!black}

\lstdefinelanguage{json}{
	basicstyle=\normalfont\ttfamily,
	numbers=left,
	numberstyle=\scriptsize,
	stepnumber=1,
	numbersep=8pt,
	showstringspaces=false,
	breaklines=true,
	frame=lines,
	backgroundcolor=\color{background},
	literate=
	*{0}{{{\color{numb}0}}}{1}
	{1}{{{\color{numb}1}}}{1}
	{2}{{{\color{numb}2}}}{1}
	{3}{{{\color{numb}3}}}{1}
	{4}{{{\color{numb}4}}}{1}
	{5}{{{\color{numb}5}}}{1}
	{6}{{{\color{numb}6}}}{1}
	{7}{{{\color{numb}7}}}{1}
	{8}{{{\color{numb}8}}}{1}
	{9}{{{\color{numb}9}}}{1}
	{:}{{{\color{punct}{:}}}}{1}
	{,}{{{\color{punct}{,}}}}{1}
	{\{}{{{\color{delim}{\{}}}}{1}
	{\}}{{{\color{delim}{\}}}}}{1}
	{[}{{{\color{delim}{[}}}}{1}
	{]}{{{\color{delim}{]}}}}{1},
}

\begin{document}
	All the endpoints can return status codes \mono{4XX}, \mono{5XX}, non \mono{20X} status codes will have no extra information, just the code. The following documentation for the endpoints are for the basic or alternative flows.	
	
	\section{GET endpoints}	
	\subsection{/genres}
	
	Query all genres registered in the database
	
	\subsubsection{Parameters}
	NONE
	
	\subsubsection{Response}
	A \mono{JSON} with the following structure:
	
\begin{lstlisting}[language=json]
{
	"genres": Array<{
		"id": number,
		"genre": string
	}>
}
\end{lstlisting}
	

\subsection{/media}
Get some media, movies, episodes or both.

\subsubsection{Parameters}
A \mono{JSON} with the following structure

\begin{lstlisting}[language=json]
{
	type_of_content: string, //required, enum: ["movie", "episode", "video"]
	search: string,
	min_rating: number,
	genre: number
}
\end{lstlisting}

If \mono{min\_rating} or \mono{genre} are undefined, default values are 0. If \mono{search} is undefined or simply '' or any value that in Javascript has a boolean value of \mono{false}, then, this field is ignored.

\subsubsection{Response}
A \mono{JSON} with the following structure

\begin{lstlisting}[language=json]
{
	"movies": Array<{
		"id": number,
		"rating": number,
		"duration": number,
		"year": number,
		"name": string,
		"cover": string | null
	}>,
	"episodes": Array<{
		"id": number,
		"serie_id": number,
		"serie_name": string,
		"n_episode": number,
		"n_season": number,
		"rating": number,
		"duration": number,
		"name": string,
		"cover": string | null
	}>
}
\end{lstlisting}

If only movies is requested, then the property "episodes" will not appear, and vice versa, if both are requested both will appear.

\subsection{/episodes}

Query all the episodes of a given serie

\subsubsection{Parameters}
A \mono{JSON} with the following structure

\begin{lstlisting}[language=json]
{
	"serie_id": number
}
\end{lstlisting}

\subsubsection{Response}

\begin{lstlisting}[language=json]
{
	"episodes": Array<{
		"id": number,
		"serie_id": number,
		"n_episode": number,
		"n_season": number,
		"rating": number,
		"duration": number,
		"name": string,
		"cover": string | null
	}>
}
\end{lstlisting}

\section{PUT endpoints}

\subsection{/media-rating}

Query all the episodes of a given serie

\subsubsection{Parameters}
A \mono{JSON} with the following structure

\begin{lstlisting}[language=json]
{
	"type_of_content": string, //required, enum: ["movie", "episode"]
	"id": number, // required
	"new_rating": number // required
}
\end{lstlisting}

\subsubsection{Response}
NONE

\end{document}