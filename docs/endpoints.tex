\newpage
Todos los endpoints que se tratarán a continuación pueden retornar status codes status codes \mono{4XX} o \mono{5XX}, los que sean \mono{20X} status codes no tendrán información extra, sólo el status code. La siguiente documentación para los endpoints aplica para el flujo básico.
	
\section{Apéndice: GET endpoints}	
\subsection{/genres}

Retorna todos los géneros registrados en la base de datos
	
\subsubsection{Parameters}
NONE
	
\subsubsection{Response}
Un \mono{JSON} con la siguiente estructura:
	
\begin{lstlisting}[language=json]
{
	"genres": Array<{
		"id": number,
		"genre": string
	}>
}
\end{lstlisting}
	

\subsection{/media}
Retorna información sobre un tipo de contenido, ya sea películas, episodios o ambos.

\subsubsection{Parameters}
Un \mono{JSON} con la siguiente estructura:

\begin{lstlisting}[language=json]
{
	type_of_content: string, //required, enum: ["movie", "episode", "video"]
	search: string,
	min_rating: number,
	genre: number
}
\end{lstlisting}


Si \mono{min\_rating} o \mono{genre} son \mono{undefined}, los valores default son 0. Si \mono{search} es \mono{undefined} o simplemente '' o cualquier otro valor que en Javascript tenga un valor booleano de \mono{false}, entonces, este campo es ignorado.

\subsubsection{Response}
Un \mono{JSON} con la siguiente estructura:

\begin{lstlisting}[language=json]
{
	"movies": Array<{
		"id": number,
		"rating": number,
		"duration": number,
		"year": number,
		"name": string,
		"cover": string | null
	}>,
	"episodes": Array<{
		"id": number,
		"serie_id": number,
		"serie_name": string,
		"n_episode": number,
		"n_season": number,
		"rating": number,
		"duration": number,
		"name": string,
		"cover": string | null
	}>
}
\end{lstlisting}

Si solamente \mono{movies} es pedido, entonces la propiedad "episodes" no aparecerá y viceversa, si los dos son pedidos, entonces los dos aparecerán.

\subsection{/episodes}
Retorna todos los episodios de una determinada serie

\subsubsection{Parameters}
Un \mono{JSON} con la siguiente estructura:

\begin{lstlisting}[language=json]
{
	"serie_id": number
}
\end{lstlisting}

\subsubsection{Response}

\begin{lstlisting}[language=json]
{
	"episodes": Array<{
		"id": number,
		"serie_id": number,
		"n_episode": number,
		"n_season": number,
		"rating": number,
		"duration": number,
		"name": string,
		"cover": string | null
	}>
}
\end{lstlisting}

\section{Apéndice: PUT endpoints}

\subsection{/media-rating}

Cambia el rating de algún episodio o película según sea el caso

\subsubsection{Parameters}
Un \mono{JSON} con la siguiente estructura:

\begin{lstlisting}[language=json]
{
	"type_of_content": string, //required, enum: ["movie", "episode"]
	"id": number, // required
	"new_rating": number // required
}
\end{lstlisting}

\subsubsection{Response}
NONE